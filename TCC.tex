%% abtex2-modelo-trabalho-academico.tex, v<VERSION> laurocesar
%% Copyright 2012-<COPYRIGHT_YEAR> by abnTeX2 group at http://www.abntex.net.br/ 
%%
%% This work may be distributed and/or modified under the
%% conditions of the LaTeX Project Public License, either version 1.3
%% of this license or (at your option) any later version.
%% The latest version of this license is in
%%   http://www.latex-project.org/lppl.txt
%% and version 1.3 or later is part of all distributions of LaTeX
%% version 2005/12/01 or later.
%%
%% This work has the LPPL maintenance status `maintained'.
%% 
%% The Current Maintainer of this work is the abnTeX2 team, led
%% by Lauro César Araujo. Further information are available on 
%% http://www.abntex.net.br/
%%
%% This work consists of the files abntex2-modelo-trabalho-academico.tex,
%% abntex2-modelo-include-comandos and abntex2-modelo-references.bib
%%

% ------------------------------------------------------------------------
% ------------------------------------------------------------------------
% abnTeX2: Modelo de Trabalho Academico (tese de doutorado, dissertacao de
% mestrado e trabalhos monograficos em geral) em conformidade com 
% ABNT NBR 14724:2011: Informacao e documentacao - Trabalhos academicos -
% Apresentacao
% ------------------------------------------------------------------------
% ------------------------------------------------------------------------

\documentclass[
	% -- opções da classe memoir --
	12pt,				% tamanho da fonte
	openright,			% capítulos começam em pág ímpar (insere página vazia caso preciso)
	twoside,			% para impressão em recto e verso. Oposto a oneside
	a4paper,			% tamanho do papel. 
	% -- opções da classe abntex2 --
	%chapter=TITLE,		% títulos de capítulos convertidos em letras maiúsculas
	%section=TITLE,		% títulos de seções convertidos em letras maiúsculas
	%subsection=TITLE,	% títulos de subseções convertidos em letras maiúsculas
	%subsubsection=TITLE,% títulos de subsubseções convertidos em letras maiúsculas
	% -- opções do pacote babel --
	english,			% idioma adicional para hifenização
	french,				% idioma adicional para hifenização
	spanish,			% idioma adicional para hifenização
	brazil				% o último idioma é o principal do documento
	]{abntex2}

% ---
% Pacotes básicos 
% ---
\usepackage{lmodern}			% Usa a fonte Latin Modern			
\usepackage[T1]{fontenc}		% Selecao de codigos de fonte.
\usepackage[utf8]{inputenc}		% Codificacao do documento (conversão automática dos acentos)
\usepackage{indentfirst}		% Indenta o primeiro parágrafo de cada seção.
\usepackage{color}				% Controle das cores
\usepackage{graphicx}			% Inclusão de gráficos
\usepackage{microtype} 			% para melhorias de justificação
% ---


% ---
% Pacotes de citações
% ---
\usepackage[brazilian,hyperpageref]{backref}	 % Paginas com as citações na bibl
\usepackage[alf]{abntex2cite}	% Citações padrão ABNT

% --- 
% CONFIGURAÇÕES DE PACOTES
% --- 


% ---
% Configurações do pacote backref
% Usado sem a opção hyperpageref de backref
\renewcommand{\backrefpagesname}{Citado na(s) página(s):~}
% Texto padrão antes do número das páginas
\renewcommand{\backref}{}
% Define os textos da citação
\renewcommand*{\backrefalt}[4]{
	\ifcase #1 %
		Nenhuma citação no texto.%
	\or
		Citado na página #2.%
	\else
		Citado #1 vezes nas páginas #2.%
	\fi}%
% ---

% ---
% Informações de dados para CAPA e FOLHA DE ROSTO
% ---
\titulo{Análise e classificação de comentários}
\autor{Fabrício Velôso de Jesus}
\local{Brasil}
\data{2018, v1.0}
\orientador{Tiago Palma Pagano}

\instituicao{%
  Universidade Federal do Recôncavo da Bahia - UFRB
  Bacharelado em Ciências Exatas e Tecnológicas}
\tipotrabalho{Monografia (Graduação)}
% O preambulo deve conter o tipo do trabalho, o objetivo, 
% o nome da instituição e a área de concentração 
\preambulo{Trabalho monografico apresentado para obtenção do grau de bacharel em ciências exatas e tecnológicas.}
% ---


% ---
% Configurações de aparência do PDF final

% alterando o aspecto da cor azul
\definecolor{blue}{RGB}{41,5,195}

% informações do PDF
\makeatletter
\hypersetup{
     	%pagebackref=true,
		pdftitle={\@title}, 
		pdfauthor={\@author},
    	pdfsubject={\imprimirpreambulo},
	    pdfcreator={LaTeX with abnTeX2},
		pdfkeywords={abnt}{latex}{abntex}{abntex2}{trabalho acadêmico}, 
		colorlinks=true,       		% false: boxed links; true: colored links
    	linkcolor=blue,          	% color of internal links
    	citecolor=blue,        		% color of links to bibliography
    	filecolor=magenta,      		% color of file links
		urlcolor=blue,
		bookmarksdepth=4
}
\makeatother
% --- 

% ---
% Posiciona figuras e tabelas no topo da página quando adicionadas sozinhas
% em um página em branco. Ver https://github.com/abntex/abntex2/issues/170
\makeatletter
\setlength{\@fptop}{5pt} % Set distance from top of page to first float
\makeatother
% ---

% ---
% Possibilita criação de Quadros e Lista de quadros.
% Ver https://github.com/abntex/abntex2/issues/176
%
\newcommand{\quadroname}{Quadro}
\newcommand{\listofquadrosname}{Lista de quadros}

\newfloat[chapter]{quadro}{loq}{\quadroname}
\newlistof{listofquadros}{loq}{\listofquadrosname}
\newlistentry{quadro}{loq}{0}

% configurações para atender às regras da ABNT
\setfloatadjustment{quadro}{\centering}
\counterwithout{quadro}{chapter}
\renewcommand{\cftquadroname}{\quadroname\space} 
\renewcommand*{\cftquadroaftersnum}{\hfill--\hfill}

\setfloatlocations{quadro}{hbtp} % Ver https://github.com/abntex/abntex2/issues/176
% ---

% --- 
% Espaçamentos entre linhas e parágrafos 
% --- 

% O tamanho do parágrafo é dado por:
\setlength{\parindent}{1.3cm}

% Controle do espaçamento entre um parágrafo e outro:
\setlength{\parskip}{0.2cm}  % tente também \onelineskip

% ---
% compila o indice
% ---
\makeindex
% ---

% ----
% Início do documento
% ----
\begin{document}

% Seleciona o idioma do documento (conforme pacotes do babel)
%\selectlanguage{english}
\selectlanguage{brazil}

% Retira espaço extra obsoleto entre as frases.
\frenchspacing 

% ----------------------------------------------------------
% ELEMENTOS PRÉ-TEXTUAIS
% ----------------------------------------------------------
% \pretextual

% ---
% Capa
% ---
\imprimircapa
% ---

% ---
% Folha de rosto
% (o * indica que haverá a ficha bibliográfica)
% ---
\imprimirfolhaderosto*
% ---

% ---
% Inserir a ficha bibliografica
% ---

% Isto é um exemplo de Ficha Catalográfica, ou ``Dados internacionais de
% catalogação-na-publicação''. Você pode utilizar este modelo como referência. 
% Porém, provavelmente a biblioteca da sua universidade lhe fornecerá um PDF
% com a ficha catalográfica definitiva após a defesa do trabalho. Quando estiver
% com o documento, salve-o como PDF no diretório do seu projeto e substitua todo
% o conteúdo de implementação deste arquivo pelo comando abaixo:
%
% \begin{fichacatalografica}
%     \includepdf{fig_ficha_catalografica.pdf}
% \end{fichacatalografica}

% ---

% ---
% RESUMOS
% ---

% resumo em português
\setlength{\absparsep}{18pt} % ajusta o espaçamento dos parágrafos do resumo
\begin{resumo}
 

 \textbf{Palavras-chave}: 
\end{resumo}

% resumo em inglês
\begin{resumo}[Abstract]
 \begin{otherlanguage*}{english}
   

   \vspace{\onelineskip}
 
   \noindent 
   \textbf{Keywords}: 
 \end{otherlanguage*}
\end{resumo}


% ---
% inserir lista de ilustrações
% ---
\pdfbookmark[0]{\listfigurename}{lof}
\listoffigures*
\cleardoublepage
% ---

% ---
% inserir lista de quadros
% ---
\pdfbookmark[0]{\listofquadrosname}{loq}
\listofquadros*
\cleardoublepage
% ---

% ---
% inserir lista de tabelas
% ---
\pdfbookmark[0]{\listtablename}{lot}
\listoftables*

\cleardoublepage
% ---

% ---
% inserir lista de abreviaturas e siglas
% ---
\begin{siglas}
  \item[IA] Sigla para Inteligência Artificial
  \item[SOM] \emph{Self-Organizing Map}, em portugês Mapas auto organizáveis
  \item[RNAs] Sigla para Redes Neurais Artificiais
  
\end{siglas}
% ---

% ---
% inserir lista de símbolos
% ---
\begin{simbolos}
  \item[$ \Gamma $] Letra grega Gama
  
\end{simbolos}
% ---

% ---
% inserir o sumario
% ---
\pdfbookmark[0]{\contentsname}{toc}
\tableofcontents*
\cleardoublepage
% ---



% ----------------------------------------------------------
% ELEMENTOS TEXTUAIS
% ----------------------------------------------------------
\textual

% ----------------------------------------------------------
% Introdução (exemplo de capítulo sem numeração, mas presente no Sumário)
% ----------------------------------------------------------
\chapter{Introdução}
% ----------------------------------------------------------

\section{Objetivo}
Analisar e classificar comentários de twitter segundo seu caráter misógino.
\section{Objetivos específicos}
Utilizar métodos capazes de classificar os comentários segundo seu caráter misógino.
Dentro deste comportamento de aversão às mulheres existem subcategorias, que devem ser declaradas e evidenciadas na classificação.

Analisar caracteristicas comuns as frases que pertecem ao mesmo grupo e determinar a ocorrência e relevância de determinadas palavras para a identificação.

Determinar se tal comportamento possui direcionamento a um usuário em específico, ou é realizado de forma a generalizar todas as mulheres.

\section{Justificativa}
Como consequência, a análise dos resultados obtidos neste trabalho poderá prover um padrão especifico referente ao comportamento de usuários misóginos no twitter.

\section{Metodologia}
Aplicar métodos de mineração de dados em textos para realizar o ajuste dos dados existentes na base. 

Utilizar aprendizado de máquina nos dados ajustados para criar uma rotina de classificação das frases.
A proposta aqui é com o auxílio de redes neurais, evidenciar dados específicos encontrados em comentários que refletem um cunho misógino, no qual destacamos o método de mapas auto organizáveis com o intuito de evidenciar características comuns em frases que possuem a mesma classificação.


\section{Problematização}
Com auxílio de métodos inerentes a inteligência artificial é possível determinar a existência de misoginia em um comentário?

Através do agrupamento de características é praticável a classificação das frases misóginas em subcategorias?

Existe um padrão para comentários que apresentam cunho misógino? 
% Capitulo de revisão de literatura
% ---
\chapter{Referêncial Teórico}
Neste capítulo as referências conceituais e  conceitos envolvidos neste trabalho serão descritos. Partindo da definição de misoginia, passando pelas técnicas envolvidas, e arrematando com as concepções de analise dos dados.
\section{Misoginia}
\subsection{Cassificação}
\section{Mineração de Textos}
\section{Inteligência Artificial}
\subsection{Redes Neurais Artificiais}
Segundo \citeonline{braga2000redes} RNAs são sistemas paralelos distribuidos compostos por unidades de processamento simples (nodos) que calculam determinadas funções matemáticas (normalmente não-lineares). Essa unidades são dispostas em uma ou mais camadas e interligadas por um grande número de conexões, geralmente unidirecionais. Estes modelos de conexões normalmente estão associados a pesos, os quais aramazenam o conhecimento representado no modelo e servem para ponderar a entrada recebida por cada neurônio da rede. O funcionamento destas redes é inspirado em uma estrutura física natural: o cérebro humano.

Por volta do fim da década de 1950, na Universidade de Cornell, Rosenblatt deu continuidade às idéias de McCulloch. Criando uma genuína rede de múltiplos neurônios do tipo \emph{discriminadores lineares} esta rede foi descrita como rede de \emph{perceptron}. Um perceptron é uma rede com a seguinte topologia, os neurônios são dispostoos em váiras \emph{camadas}. Os que recebem das entradas diretamente formam o que é chamada de \emph{camada de entrada}. A camada que recebe a saída da camada de entrada como entrada constituem a segunda camada e assim consecutivamente até a ultima camada que é chamada de \emph{camada de saída}. Camadas que ficam entre as de entrada e saída são comumente referidas como \emph{camadas ocultas}.

Com referência à \autoref{fig_rede_multicamada}.Uma rede neural multicamada de \emph{K} camadas, terá como entrada um vetor \textbf{x} de dimensão $J_0$ de componentes $x_{j_0}, j_0 = 1,2, ... J_0$. Estas conectam-se às entradas dos $J_1$ neurônios numa primeira camada. As saídas $u_lj_1,j_1 = 1,2, ... J_1$ destes, formando as componentes de um novo vetor \textbf{u$_1$} de dimensão $J_1$, conectam-se às entradas dos $J_2$ neurônios da camada seguinte e assim sucessivamente até a última camada que consistirá de $J_K$ neurônios fornecendo como saída da rede um vetor \textbf{y = u$_K$} de dimensão $J_K$. Genéricamente, $u_{kj_k}$ denota a saída do $j_k$ -ésima entrada da rede, e para $k=K$ a $j_k$ -ésima saída da rede.\cite[p. 39--40]{kovacs2002redes}

\begin{figure}[htb]
	\caption{\label{fig_rede_multicamada}Rede Neural Multicamada}
	\begin{center}
	    \includegraphics[scale=0.7]{imagens/rede_neural_multicamada.pdf}
	\end{center}
	\legend{Fonte: \citeonline[p. 40]{kovacs2002redes}}
\end{figure}

Ainda segundo \citeonline{kovacs2002redes} e \citeonline{braga2000redes} o problema que Rosenblatt propôs a resolver foi o de casos simples com implementação de funções booleanas \textbf{E} e \textbf{OU} de duas variáveis, que são problemas linearmente separáveis, isto é, problemas cuja solução pode ser obtida ao dividir o espaço de entrada em duas regiões através de uma reta. O perceptron, não consegue detectar conectividade, paridade e simetria, que são problemas não-linearmente separáveis. Estes são exemplos de \textit{hard learning problems} (problemas difíceis de aprender).

A abordagem conexionista ficou adormecida durante os anos 70, porém alguns pesquisadores continuaram desenvolvendo trabalhos na área. Dentre eles podem ser citados Igor Aleksander (redes sem pesos) na Inglaterra, Kunihiko Fukushima (cognitron e neocognitron) no Japão, Steven Grossberg (sistemas auto-adaptativos) nos EUA, e Teuvo Kohonen (memórias associativas e auto-organizadas) na Finlândia.
\subsubsection{Motivação para as RNAs: redes biológicas}
O cérebrohumano é um imenso e complexo bosque de células e conexões intercelulares. Esse bosque emaranhado é composto de aproximadamente 100 bilhões de neurônios ($ 1 * 10^{11}$) de formas e tamanhos diferentes. Considera-se que apenas no córtex cerebral, que contém quase a metade desse número, isto é, cerca de 50 bilhões, existam mais de 500 tipos de neurônios morfologicamente diferentes, distribuídos em 52 áreas.\cite[p.18]{mora2016continuum} 

A estrutura dos nodos, a topologia dessas conexões e o comportamento conjunto dos neurônios naturais constroem a base de estudo das RNAs. As RNAs tendem a reproduzir as funções das redes biológicas, buscando colocar em prática a sua dinâmica e seu comportamento básico. 

Conforme \citeonline{braga2000redes}, como caracteristicas comuns, ambos os sistemas são baseados em unidades de computação paralela e distribuída que se comunicam por meio de conexões sinápticas, possuem detetores de características, redundância e modularização das conexões. Apesar de pouca similaridade entre os dois sistemas do ponto de vista biológico, estas características semelhantes permitem às RNAs reproduzirem com fidelidade várias funções inerentes dos seres humanos
\subsection{Mapas Auto Organizáveis de Kohonen}
\chapter{Desenvolvimento}
\chapter{Testes e Análise de Resultados}
\chapter{Conclusão}
% ---
% ----------------------------------------------------------
% ELEMENTOS PÓS-TEXTUAIS
% ----------------------------------------------------------
\postextual
% ----------------------------------------------------------
% ----------------------------------------------------------
% Referências bibliográficas
% ----------------------------------------------------------
\bibliography{Referências}

% ----------------------------------------------------------
% Glossário
% ----------------------------------------------------------
%
% Consulte o manual da classe abntex2 para orientações sobre o glossário.
%
%\glossary

% ----------------------------------------------------------
% Apêndices
% ----------------------------------------------------------

% ---
% Inicia os apêndices
% ---

% Imprime uma página indicando o início dos apêndices
\partapendices

% ----------------------------------------------------------

%---------------------------------------------------------------------
% INDICE REMISSIVO
%---------------------------------------------------------------------
\phantompart
\printindex
%---------------------------------------------------------------------

\end{document}
