%% abtex2-modelo-trabalho-academico.tex, v<VERSION> laurocesar
%% Copyright 2012-<COPYRIGHT_YEAR> by abnTeX2 group at http://www.abntex.net.br/ 
%%
%% This work may be distributed and/or modified under the
%% conditions of the LaTeX Project Public License, either version 1.3
%% of this license or (at your option) any later version.
%% The latest version of this license is in
%%   http://www.latex-project.org/lppl.txt
%% and version 1.3 or later is part of all distributions of LaTeX
%% version 2005/12/01 or later.
%%
%% This work has the LPPL maintenance status `maintained'.
%% 
%% The Current Maintainer of this work is the abnTeX2 team, led
%% by Lauro César Araujo. Further information are available on 
%% http://www.abntex.net.br/
%%
%% This work consists of the files abntex2-modelo-trabalho-academico.tex,
%% abntex2-modelo-include-comandos and abntex2-modelo-references.bib
%%

% ------------------------------------------------------------------------
% ------------------------------------------------------------------------
% abnTeX2: Modelo de Trabalho Academico (tese de doutorado, dissertacao de
% mestrado e trabalhos monograficos em geral) em conformidade com 
% ABNT NBR 14724:2011: Informacao e documentacao - Trabalhos academicos -
% Apresentacao
% ------------------------------------------------------------------------
% ------------------------------------------------------------------------

\documentclass[
	% -- opções da classe memoir --
	12pt,				% tamanho da fonte
	openright,			% capítulos começam em pág ímpar (insere página vazia caso preciso)
	twoside,			% para impressão em recto e verso. Oposto a oneside
	a4paper,			% tamanho do papel. 
	% -- opções da classe abntex2 --
	%chapter=TITLE,		% títulos de capítulos convertidos em letras maiúsculas
	%section=TITLE,		% títulos de seções convertidos em letras maiúsculas
	%subsection=TITLE,	% títulos de subseções convertidos em letras maiúsculas
	%subsubsection=TITLE,% títulos de subsubseções convertidos em letras maiúsculas
	% -- opções do pacote babel --
	english,			% idioma adicional para hifenização
	french,				% idioma adicional para hifenização
	spanish,			% idioma adicional para hifenização
	brazil				% o último idioma é o principal do documento
	]{abntex2}

% ---
% Pacotes básicos 
% ---
\usepackage{lmodern}			% Usa a fonte Latin Modern			
\usepackage[T1]{fontenc}		% Selecao de codigos de fonte.
\usepackage[utf8]{inputenc}		% Codificacao do documento (conversão automática dos acentos)
\usepackage{indentfirst}		% Indenta o primeiro parágrafo de cada seção.
\usepackage{color}				% Controle das cores
\usepackage{graphicx}			% Inclusão de gráficos
\usepackage{microtype} 			% para melhorias de justificação
\usepackage{enumerate}			% para numeração das listas 
% ---


% ---
% Pacotes de citações
% ---
\usepackage[brazilian,hyperpageref]{backref}	 % Paginas com as citações na bibl
\usepackage[alf]{abntex2cite}	% Citações padrão ABNT

% --- 
% CONFIGURAÇÕES DE PACOTES
% --- 


% ---
% Configurações do pacote backref
% Usado sem a opção hyperpageref de backref
\renewcommand{\backrefpagesname}{Citado na(s) página(s):~}
% Texto padrão antes do número das páginas
\renewcommand{\backref}{}
% Define os textos da citação
\renewcommand*{\backrefalt}[4]{
	\ifcase #1 %
		Nenhuma citação no texto.%
	\or
		Citado na página #2.%
	\else
		Citado #1 vezes nas páginas #2.%
	\fi}%
% ---

% ---
% Informações de dados para CAPA e FOLHA DE ROSTO
% ---
\titulo{Análise e classificação de comentários}
\autor{Fabrício Velôso de Jesus}
\local{Brasil}
\data{2018, v1.0}
\orientador{Tiago Palma Pagano}

\instituicao{%
  Universidade Federal do Recôncavo da Bahia - UFRB
  Bacharelado em Ciências Exatas e Tecnológicas}
\tipotrabalho{Monografia (Graduação)}
% O preambulo deve conter o tipo do trabalho, o objetivo, 
% o nome da instituição e a área de concentração 
\preambulo{Trabalho monografico apresentado para obtenção do grau de bacharel em ciências exatas e tecnológicas.}
% ---


% ---
% Configurações de aparência do PDF final

% alterando o aspecto da cor azul
\definecolor{blue}{RGB}{41,5,195}

% informações do PDF
\makeatletter
\hypersetup{
     	%pagebackref=true,
		pdftitle={\@title}, 
		pdfauthor={\@author},
    	pdfsubject={\imprimirpreambulo},
	    pdfcreator={LaTeX with abnTeX2},
		pdfkeywords={abnt}{latex}{abntex}{abntex2}{trabalho acadêmico}, 
		colorlinks=true,       		% false: boxed links; true: colored links
    	linkcolor=blue,          	% color of internal links
    	citecolor=blue,        		% color of links to bibliography
    	filecolor=magenta,      		% color of file links
		urlcolor=blue,
		bookmarksdepth=4
}
\makeatother
% --- 

% ---
% Posiciona figuras e tabelas no topo da página quando adicionadas sozinhas
% em um página em branco. Ver https://github.com/abntex/abntex2/issues/170
\makeatletter
\setlength{\@fptop}{5pt} % Set distance from top of page to first float
\makeatother
% ---

% ---
% Possibilita criação de Quadros e Lista de quadros.
% Ver https://github.com/abntex/abntex2/issues/176
%
\newcommand{\quadroname}{Quadro}
\newcommand{\listofquadrosname}{Lista de quadros}

\newfloat[chapter]{quadro}{loq}{\quadroname}
\newlistof{listofquadros}{loq}{\listofquadrosname}
\newlistentry{quadro}{loq}{0}

% configurações para atender às regras da ABNT
\setfloatadjustment{quadro}{\centering}
\counterwithout{quadro}{chapter}
\renewcommand{\cftquadroname}{\quadroname\space} 
\renewcommand*{\cftquadroaftersnum}{\hfill--\hfill}

\setfloatlocations{quadro}{hbtp} % Ver https://github.com/abntex/abntex2/issues/176
% ---

% --- 
% Espaçamentos entre linhas e parágrafos 
% --- 

% O tamanho do parágrafo é dado por:
\setlength{\parindent}{1.3cm}

% Controle do espaçamento entre um parágrafo e outro:
\setlength{\parskip}{0.2cm}  % tente também \onelineskip

% ---
% compila o indice
% ---
\makeindex
% ---

% ----
% Início do documento
% ----
\begin{document}

% Seleciona o idioma do documento (conforme pacotes do babel)
%\selectlanguage{english}
\selectlanguage{brazil}

% Retira espaço extra obsoleto entre as frases.
\frenchspacing 

% ----------------------------------------------------------
% ELEMENTOS PRÉ-TEXTUAIS
% ----------------------------------------------------------
% \pretextual

% ---
% Capa
% ---
\imprimircapa
% ---

% ---
% Folha de rosto
% (o * indica que haverá a ficha bibliográfica)
% ---
\imprimirfolhaderosto*
% ---

% ---
% Inserir a ficha bibliografica
% ---

% Isto é um exemplo de Ficha Catalográfica, ou ``Dados internacionais de
% catalogação-na-publicação''. Você pode utilizar este modelo como referência. 
% Porém, provavelmente a biblioteca da sua universidade lhe fornecerá um PDF
% com a ficha catalográfica definitiva após a defesa do trabalho. Quando estiver
% com o documento, salve-o como PDF no diretório do seu projeto e substitua todo
% o conteúdo de implementação deste arquivo pelo comando abaixo:
%
% \begin{fichacatalografica}
%     \includepdf{fig_ficha_catalografica.pdf}
% \end{fichacatalografica}

% ---

% ---
% RESUMOS
% ---

% resumo em português
\setlength{\absparsep}{18pt} % ajusta o espaçamento dos parágrafos do resumo
\begin{resumo}
 

 \textbf{Palavras-chave}: 
\end{resumo}

% resumo em inglês
\begin{resumo}[Abstract]
 \begin{otherlanguage*}{english}
   

   \vspace{\onelineskip}
 
   \noindent 
   \textbf{Keywords}: 
 \end{otherlanguage*}
\end{resumo}


% ---
% inserir lista de ilustrações
% ---
\pdfbookmark[0]{\listfigurename}{lof}
\listoffigures*
\cleardoublepage
% ---

% ---
% inserir lista de quadros
% ---
\pdfbookmark[0]{\listofquadrosname}{loq}
\listofquadros*
\cleardoublepage
% ---

% ---
% inserir lista de tabelas
% ---
\pdfbookmark[0]{\listtablename}{lot}
\listoftables*

\cleardoublepage
% ---

% ---
% inserir lista de abreviaturas e siglas
% ---
\begin{siglas}
  \item[IA] Sigla para Inteligência Artificial
  \item[KDD] \emph{Knowledge Discovery in Database}, em portugês Descoberta de Conhecimento em Bases de Dados
  \item[SOM] \emph{Self-Organizing Map}, em portugês Mapas auto organizáveis
  \item[RNAs] Sigla para Redes Neurais Artificiais
  
\end{siglas}
% ---

% ---
% inserir lista de símbolos
% ---
\begin{simbolos}
  \item[$ \Gamma $] Letra grega Gama
  
\end{simbolos}
% ---

% ---
% inserir o sumario
% ---
\pdfbookmark[0]{\contentsname}{toc}
\tableofcontents*
\cleardoublepage
% ---



% ----------------------------------------------------------
% ELEMENTOS TEXTUAIS
% ----------------------------------------------------------
\textual

% ----------------------------------------------------------
% Introdução (exemplo de capítulo sem numeração, mas presente no Sumário)
% ----------------------------------------------------------
\chapter{Introdução}
% ----------------------------------------------------------

\section{Objetivo}
Analisar e classificar comentários de twitter segundo seu caráter misógino.
\section{Objetivos específicos}
Utilizar métodos capazes de classificar os comentários segundo seu caráter misógino.
Dentro deste comportamento de aversão às mulheres existem subcategorias, que devem ser declaradas e evidenciadas na classificação.

Analisar caracteristicas comuns as frases que pertecem ao mesmo grupo e determinar a ocorrência e relevância de determinadas palavras para a identificação.

Determinar se tal comportamento possui direcionamento a um usuário em específico, ou é realizado de forma a generalizar todas as mulheres.

\section{Justificativa}
Como consequência, a análise dos resultados obtidos neste trabalho poderá prover um padrão especifico referente ao comportamento de usuários misóginos no twitter.

\section{Metodologia}
Aplicar métodos de mineração de dados em textos para realizar o ajuste dos dados existentes na base. 

Utilizar aprendizado de máquina nos dados ajustados para criar uma rotina de classificação das frases.
A proposta aqui é com o auxílio de redes neurais, evidenciar dados específicos encontrados em comentários que refletem um cunho misógino, no qual destacamos o método de mapas auto organizáveis com o intuito de evidenciar características comuns em frases que possuem a mesma classificação.


\section{Problematização}
Com auxílio de métodos inerentes a inteligência artificial é possível determinar a existência de misoginia em um comentário?

Através do agrupamento de características é praticável a classificação das frases misóginas em subcategorias?

Existe um padrão para comentários que apresentam cunho misógino? 
% Capitulo de revisão de literatura
% ---
\chapter{Referêncial Teórico}
Neste capítulo as referências conceituais e  conceitos envolvidos neste trabalho serão descritos. Partindo da definição de misoginia, passando pelas técnicas envolvidas, e arrematando com as concepções de analise dos dados.

\section{Misoginia}
\subsection{Classificação}
\section{Mineração de Dados}
Conforme \citeonline{amorim2006conceitos} e  \citeonline{santos2008computaccao} a definição de mineração de dados (\emph{Data Mining}) pode ser descrita como o conjunto de técnicas que permite a extração de conhecimentos, padrões e relações de grandes massas de dados que não seriam descobertas com facilidade a olho nu pelo homem.

A descoberta de padrões constitui-se de um processo que se inicia pela escolha dos dados que documentam de alguma maneira a pergunta que o especialista deseja responder. Os dados são integrados e pré-processados para que sejam entregues estruturados, higienizados, selecionados e padronizados à tarefa de mineração de dados. Na tarefa de mineração aplica-se alguma técnica inteligente capaz de encontrar soluções que auxiliam o especialista na descoberta de uma resposta. O resultado desta tarefa deve ser pós-processado para que se apresentem análises qualitativas e/ou quantitativa dos elementos encontrados e, quando possível, apresentados de maneira que possa ser interpretado de maneira a facilitar a tomada de decisão. \cite[p.569 - p.570]{inproceedings}

Todo este processo citado acima, pode ser chamado de Descoberta de Conhecimento em Base de Dados ou simplismente KDD, consoante \apudonline{fayyad1996kdd}{inproceedings}.

Uma das definições mais utilizadas para o termo KDD é a de Fayyad, que o define como "um processo não trivial de identificação de novos padrões válidos, úteis e compreensíveis".\cite[p.3]{camilo2009mineraccao}

Ainda segundo \citeonline{camilo2009mineraccao}, até agora não é consenso a definição dos termos \emph{Data Mining} e \emph{KDD}. No entanto, todos concordam que o processo de mineração deve ser interativo, iterativo e particionado em fases, comforme visto na \autoref{fig_processo_descoberta}. 

\begin{figure}[htb]
	\caption{\label{fig_processo_descoberta}Processo de Descoberta de Conhecimento em Base de Dados}
	\begin{center}
	    \includegraphics[scale=0.4]{imagens/Processo_de_descoberta_de_conhecimento_em_base_de_dados.pdf}
	\end{center}
	\legend{Fonte: \citeonline[p.570]{inproceedings}}
\end{figure}

\subsection{Tipos de estrutura de dados}
Segundo \citeonline{sargiani2018identificaccao} a estrutura com a qual os dados são apresentados é importante, ela induz de forma direta nas ferramentas que serão utilizadas e nas técnicas de tratamento. Na análise de dados  não estruturados ferramentas que permitem extração de conhecimento a partir de dados sem estrutura são utilizadas. Para dados semi-estruturados as técnicas são definidas  com base no caso em específico. E para dados estruturados bancos relacionais são utilizados.

De acordo com \citeonline{morais2007mineraccao} e \citeonline{sargiani2018identificaccao}, o processo de descoberta de conhecimento em \textbf{dados estruturados} é feito através do uso de ferramentas baseadas em métodos estatísticos, métodos provenientes da área de recuperação de informações e  ontologias, estes dados possuem um formato definido através de algum critério. A informação pode ser representada em \emph{datasets}, tabelas, arquivos multimídia ou arquivos texto. 

Os \textbf{dados semi-estruturados} não possuem um formato adequado para o uso de apenas uma ontologia, porém podem ser identificados, pois possuem algum grau de regularidade.\apud{barros2008hidden}{sargiani2018identificaccao}

Conforme \citeonline{inbook} e \citeonline{sargiani2018identificaccao} os chamados \textbf{dados não estruturados} ocorrem quando a informação não possui nenhum formato reconhecível, comumente correlacionado a linguagem natural, o Twitter, e-mails e conteúdo em fóruns são alguns exemplos, as informações não estão dispostas em tabelas númericas organizadas em linhas e colunas e não possuem um formato adequado para o uso de uma ontologia. A mineração de textos é uma técnica usada pelos sistemas inteligentes que engloba o processamento de dados não estruturados do tipo texto, em outras palavras as \emph{strings} ou sequência de caracteres de um texto.

\section{Mineração de Textos}
A mineração de textos consiste em extrair regularidades , padrões ou tendências para determinados objetivos, essa obtenção de informação é feita em grandes volumes de textos em linguagem natural. É um campo novo e multidisciplinar que inclui conhecimento de áreas como Estatística, Informática, Linguistica e Ciência Cognitiva. \citeonline{aranha2006tecnologia}

Conforme \citeonline{sargiani2018identificaccao} para a execução da análise é necessário estruturar esse tipo de dados não estruturados por meio de um modelo de representação, que transforma os termos de cada publicação em um valor de relevância. A análise de dados não estruturados é feita em três fases distintas.

\begin{itemize}
	\item A primeira fase do processo é a construção do vacábulário que se dá pelo processo de mineração de textos. Isto é, o texto com todos os comentários selecionados representa o \emph{corpus} inicial. Para a geração da representação (\emph{corpus} representado) é necessário seguir os seguintes passos, como descrito por \apudonline{goker2009information}{sargiani2018identificaccao} e \apudonline{da2017introduccao}{sargiani2018identificaccao}:
	\begin{enumerate}
		\item \emph{Tokenization}: A partir do caractere espaço, os comandos das instruções são separados em tokens. Os caracteres especiais como vírgulas (","), e pontuação em geral, são removidos, assim como números. Padronização de capitalização para minúsculas(ou maiúsculas) também é feita nesta fase;
		\item \emph{Stopwords}: Palavras como artigos, advérbios, pronomes, preposições, que são comuns em diferentes contextos, são removidos do processo;
		\item \emph{Stemming}: As palavras resultantes das etapas anteriores passam por uma normalização ortográfica para que sejam reduzidas ao radical. Este processo é importante, pois permite que palavras com o mesmo radical sejam consideradas como semelhantes.
	\end{enumerate}
	\item A segunda fase é a geração do \emph{corpus}. Este \emph{corpus} é uma matriz contendo todos os documentos analisados, todos os termos encontrados, e suas respectivas quantidades em cada documento;
	\item A última fase é a geração da matriz de frequências, momento em que é feita a relação entre cada documento e os termos constantes. O formato ideal a ser escolhido depende principalmente da análise que será feita posteriormente, pois o formato dessa matriz afeta diretamente no processo de análise.
\end{itemize}

\section{Inteligência Artificial}
Segundo \citeonline{da2005inteligencia} a inteligência artificial é a parte da Ciência da Computação voltada para o desenvolvimento de sistemas de computadores inteligentes, isto é, sistemas que exibem características que estão associadas à inteligência no comportamento humano, como compreensão da linguagem, aprendizado, raciocínio, resolução de problemas, entre outros.

De acordo com \citeonline{hodges1999turing} o \textbf{teste de Turing}, proposto por Alan Turing(1950), fornece uma definição operacional satisfatória de inteligência. Seu objetivo é descobrir se uma IA é inteligente a ponto de enganar um humano, de forma que ele acredite que uma pessoa está respondendo suas perguntas feitas e respondidas através de textos. O argumento de Turing é simplesmente o de que o cérebro deve também ser considerado uma máquina de estado discreto e que as únicas características do cérebro relevantes para o pensamento ou a inteligência são aquelas situadas no nível de descrição da máquina de estado discreto, portanto a materialização física é irrelevante.

Para que uma IA passe no teste de Turing, ela deve apresentar as seguintes capacidades, como descrito por \citeonline{russell2013inteligencia}
\begin{itemize}
	\item \textbf{processamento de linguagem natural} para permitir que ele se cominique com sucesso em um idioma natural;
	\item \textbf{raciocínio automatizado} para usar as informações armazenadas com a finalidade de responder a perguntas e tirar novas conclusões;
	\item \textbf{representação de conhecimento} para armazenar o que sabe ou ouve;
	\item \textbf{aprendizado de máquina} para se adaptar a novas circunstâncias e para detectar e extrapolar padrões.
\end{itemize}
Conforme \citeonline{russell2013inteligencia} o primeiro trabalho reconhecido como IA foi proposto por Warren McCulloch e Walter Pitts (1943). Este trabalho foi baseado em três fontes: o conhecimento da fisioloia básica e da função dos neurônios no cérebro; a teoria da computação de Turing; e uma análise formal da lógica proposicional criado por Russell e Whitehead. Esses pesquisadores propuseram um modelo de neurônios artificiais, onde cada neurônio se caracteriza por estar "ligado" ou "desligado", com a troca para "ligado" ocorrendo em resposta à estimulação por um número suficiente de neurônios vizinhos. O estado era considerado "equivalente em termos concretos a uma proposição que definia seu estimulo adequado". Eles mostraram que qualquer função computável podia ser calculada por certa rede de neurônios conectados e que todos os conectivos lógicos podiam ser implementados por estruturas de redes simples.

Vários trabalhos que podem ser caracterizados como IA surgiram, mas a visão proposta por Alan Tuting foi talvez a mais influente. Em 1947, ele proferia palestras sobre o tema na Sociedade Matemática de Londres e articulou um programa de trabalhos persuasivo em seu artigo de 1950, "computing Machinery and Intelligence" \citeonline{hodges1999turing}. Artigo no qual apresentou o teste de Turing, algoritmos genéticos, aprendizagem de máquina e aprendizagem por reforço.

Ainda segundo \citeonline{russell2013inteligencia} os pesquisadores da IA possuiam prognósticos ousados de seus sucessos futuros, porém entre 1966 e 1973 alguns tipos de dificuldades surgiram: 
\begin{enumerate}
	\item Primeiro tipo de dificuldade surgiu porque a maioria dos primeiros programas não tinha conhecimento de seu assunto, isto é, eles obtiam sucesso por meio de manipulações sintáticas simples;
	\item O segundo tipo de dificuldade foi a impossibilidade de tratar muitos problemas que a IA estava tentando resolver, a maior parte dos primeiros programas de IA resolvia problemas experimentando diferentes combinações de passos até encontrar a solução.\emph{O fato de um programa poder encontrar uma solução em princípio não significa que o programa contenha quaisquer dos mecanismos necessários para encontrá-la na prótica};
	\item Uma terceira dificuldade surgiu devido a algumas limitações fundamentais nas estruturas básicas que estavam sendo utilizadas para gerar a comportamento inteligente.
\end{enumerate}  

Os chamados modelos \textbf{conexionistas} para sistemas inteligentes eram vistos por alguns como concorrentes diretos dos modelos simbólicos promovidos por Newell e Simon e da abordagem logicista de McCarthy e outros pesquisadores \apudonline{smolensky1988connectionism}{russell2013inteligencia}.%duvidade nessa citação

Pode parecer óbvio que, em certo nível, os seres humanos manipulam símbolos, mas os conexionistas mais fervorosos questionavam se a manipulação de símbolos tinha qualquer função explicativa real em modelos detalhados de cognição. Essa pergunta permanece sem resposta, mas a visão atual é de que as abordagens conexionista e simbólica são complementares, e não concorrentes. Como ocorreu com a separação da IA e da ciência cognitiva, a pesquisa moderna de rede neural se bifurcou em dois campos, um preocupado com a criação de algoritmos e arquiteturas de rede eficazes e a compreensão de suas propriedades matemáticas, o outro preocupado com a modelagem cuidadosa das propriedades empíricas de neurônios reais e conjuntos de neurônios.\cite{russell2013inteligencia}

\section{Redes Neurais Artificiais}
Segundo \citeonline{braga2000redes} RNAs são sistemas paralelos distribuidos compostos por unidades de processamento simples (nodos) que calculam determinadas funções matemáticas (normalmente não-lineares). Essa unidades são dispostas em uma ou mais camadas e interligadas por um grande número de conexões, geralmente unidirecionais. Estes modelos de conexões normalmente estão associados a pesos, os quais aramazenam o conhecimento representado no modelo e servem para ponderar a entrada recebida por cada neurônio da rede. O funcionamento destas redes é inspirado em uma estrutura física natural: o cérebro humano.

Comforme \citeonline{braga2000redes} e \citeonline{kovacs2002redes}, por volta do fim da década de 1950, na Universidade de Cornell, Rosenblatt deu continuidade às idéias de McCulloch. Criando uma genuína rede de múltiplos neurônios do tipo \emph{discriminadores lineares} esta rede foi descrita como rede de \emph{perceptron}. Um perceptron é uma rede com a seguinte topologia, os neurônios são dispostoos em váiras \emph{camadas}. Os que recebem das entradas diretamente formam o que é chamada de \emph{camada de entrada}. A camada que recebe a saída da camada de entrada como entrada constituem a segunda camada e assim consecutivamente até a ultima camada que é chamada de \emph{camada de saída}. Camadas que ficam entre as de entrada e saída são comumente referidas como \emph{camadas ocultas}.

Com referência à \autoref{fig_rede_multicamada}.Uma rede neural multicamada de \emph{K} camadas, terá como entrada um vetor \textbf{x} de dimensão $J_0$ de componentes $x_{j_0}, j_0 = 1,2, ... J_0$. Estas conectam-se às entradas dos $J_1$ neurônios numa primeira camada. As saídas $u_lj_1,j_1 = 1,2, ... J_1$ destes, formando as componentes de um novo vetor \textbf{u$_1$} de dimensão $J_1$, conectam-se às entradas dos $J_2$ neurônios da camada seguinte e assim sucessivamente até a última camada que consistirá de $J_K$ neurônios fornecendo como saída da rede um vetor \textbf{y = u$_K$} de dimensão $J_K$. Genéricamente, $u_{kj_k}$ denota a saída do $j_k$ -ésima entrada da rede, e para $k=K$ a $j_k$ -ésima saída da rede.\cite[p. 39--40]{kovacs2002redes}

\begin{figure}[htb]
	\caption{\label{fig_rede_multicamada}Rede Neural Multicamada}
	\begin{center}
	    \includegraphics[scale=0.7]{imagens/rede_neural_multicamada.pdf}
	\end{center}
	\legend{Fonte: \citeonline[p.40]{kovacs2002redes}}
\end{figure}

Ainda segundo \citeonline{kovacs2002redes} e \citeonline{braga2000redes} o problema que Rosenblatt propôs a resolver foi o de casos simples com implementação de funções booleanas \textbf{E} e \textbf{OU} de duas variáveis, que são problemas linearmente separáveis, isto é, problemas cuja solução pode ser obtida ao dividir o espaço de entrada em duas regiões através de uma reta. O perceptron, não consegue detectar conectividade, paridade e simetria, que são problemas não-linearmente separáveis. Estes são exemplos de \textit{hard learning problems} (problemas difíceis de aprender).

A abordagem conexionista ficou adormecida durante os anos 70, porém alguns pesquisadores continuaram desenvolvendo trabalhos na área. Dentre eles podem ser citados Igor Aleksander (redes sem pesos) na Inglaterra, Kunihiko Fukushima (cognitron e neocognitron) no Japão, Steven Grossberg (sistemas auto-adaptativos) nos EUA, e Teuvo Kohonen (memórias associativas e auto-organizadas) na Finlândia.

\subsection{Motivação para as RNAs: redes biológicas}
O cérebro humano é um imenso e complexo bosque de células e conexões intercelulares. Esse bosque emaranhado é composto de aproximadamente 100 bilhões de neurônios ($ 1 * 10^{11}$) de formas e tamanhos diferentes. Considera-se que apenas no córtex cerebral, que contém quase a metade desse número, isto é, cerca de 50 bilhões, existam mais de 500 tipos de neurônios morfologicamente diferentes, distribuídos em 52 áreas.\cite[p.18]{mora2016continuum} 

A estrutura dos nodos, a topologia dessas conexões e o comportamento conjunto dos neurônios naturais constroem a base de estudo das RNAs. As RNAs tendem a reproduzir as funções das redes biológicas, buscando colocar em prática a sua dinâmica e seu comportamento básico. 

Conforme \citeonline{braga2000redes}, como caracteristicas comuns, ambos os sistemas são baseados em unidades de computação paralela e distribuída que se comunicam por meio de conexões sinápticas, possuem detetores de características, redundância e modularização das conexões. Apesar de pouca similaridade entre os dois sistemas do ponto de vista biológico, estas características semelhantes permitem às RNAs reproduzirem com fidelidade várias funções inerentes dos seres humanos
\section{Mapas Auto Organizáveis de Kohonen}

\chapter{Desenvolvimento}
\chapter{Testes e Análise de Resultados}
\chapter{Conclusão}
% ---
% ----------------------------------------------------------
% ELEMENTOS PÓS-TEXTUAIS
% ----------------------------------------------------------
\postextual
% ----------------------------------------------------------
% ----------------------------------------------------------
% Referências bibliográficas
% ----------------------------------------------------------
\bibliography{Referências}

% ----------------------------------------------------------
% Glossário
% ----------------------------------------------------------
%
% Consulte o manual da classe abntex2 para orientações sobre o glossário.
%
%\glossary

% ----------------------------------------------------------
% Apêndices
% ----------------------------------------------------------

% ---
% Inicia os apêndices
% ---

% Imprime uma página indicando o início dos apêndices
\partapendices

% ----------------------------------------------------------

%---------------------------------------------------------------------
% INDICE REMISSIVO
%---------------------------------------------------------------------
\phantompart
\printindex
%---------------------------------------------------------------------

\end{document}
